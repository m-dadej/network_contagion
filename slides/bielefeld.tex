\documentclass{beamer}
\usepackage{algorithm}
\usepackage{algpseudocode}
\usepackage{graphicx}
\usepackage{amsmath}
\usepackage{hyperref}
\usetheme{AnnArbor}

\title{Risk aversion heterogeneity and contagion in
endogenous financial networks}
\bibliographystyle{apalike}
\author{Mateusz Dadej}
% write an affiliation

\date{05-07-2023} % check the date of the talk

\begin{document}

\begin{frame}
\titlepage
\end{frame}

\begin{frame}{Bank's balance sheet}

    \begin{itemize}
        \item The model build on works of Aldasoro et. al (2017) and Bluhm and Krahnen (2014). %\citet{bluhm} and \citet{aldasoro}
        \item It consists of $N$ banks, each with the balance sheet satisfying following identity:
        
        \[c_i + \sum_{j=1}^{N}l_{i,j} + e_i = d_i + \sum_{j=1}^{N}b_{i,j} + q_i\]

        \item Where asset side consists of cash $c$, external assets $e$ and interbank loans $l_{i,j}$. On the other side there are deposits $d$, interbank liabilities $l_{i,j}$ and equity $q_i$.
        
        % I guess we can set only $l_{i,j}$, and just reverse the indexation for the RHS
    \end{itemize}

\end{frame}

\begin{frame}{Financial network representation}

    \begin{itemize}
        \item The double indexation represents the interconectedness among banks. $l_{i,j}$ is the value of debt claim from bank $i$ to bank $j$.
        \item Therefore, $\sum_{j=1}^{N}l_{i,j}$ is a portfolio of claims of bank $i$
        \item These assets are not necessarily limited to interbank loans, but can also include other contracts that posses credit risk.
    \end{itemize}   

\end{frame}

\begin{frame}{Objective of banks}

    \begin{itemize}
        \item The profit function of banks depend on external and interbank assets and their return, respectively $r^e$ and $r^l$. As well as the cost of its funding, that depends on the default probability $\delta$ and loss given default $\zeta$.
        
        \[\pi_i = r_{i}^e e_i + r^l l_i - (\frac{1}{1 - \zeta \delta}) r^l b_i\]

        \item The utility function of a bank is a standard CRRA function. With following expected value (obtained from taylor expansion):
        
        \[\mathbf{E}[U(\pi, \sigma_i, \sigma_\pi)] = \frac{\pi^{1-\sigma_i}}{1 - \sigma_i} - \frac{\sigma_i}{2} \pi_i^{-1-\sigma_i} \sigma_\pi\]

        \item Where $\sigma_i$ is the risk aversion of bank $i$ and $\sigma_\pi$ is the variance of the profit of bank $i$, give with:
        
        \[ e^2 * \sigma_{r_i^e} - (b_i * r_l)^2 * \zeta^2 * (1 - (\zeta * \mathbf{\delta}))^(-4) * \sigma_\delta\]

    \end{itemize}
    
\end{frame}


% frame with references

\begin{frame}{References}
    \begin{itemize}
        \item  Aldasoro, Iñaki and Delli Gatti, Domenico and Faia, Ester. "Bank networks: Contagion, systemic risk and prudential policy". Journal of Economic Behavior \& Organization, 2017, vol. 142, issue C, 164-188
        \item Bluhm, Marcel, and Jan-Pieter Krahnen. "Systemic risk in an interconnected banking system with endogenous asset markets." Journal of Financial Stability 13 (2014): 75-94.
    \end{itemize}

\end{frame}
    
\end{document}

