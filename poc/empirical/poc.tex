\documentclass{article}
\usepackage{algorithm}
\usepackage{algpseudocode}
\usepackage{natbib}
\usepackage{graphicx}
\usepackage{amsmath}
\usepackage{hyperref}
\hypersetup{
colorlinks=false,
pdfborder={0 0 0},
}
\title{Empirical evidence of 'Robust-yet-fragile' property of financial markets}
\bibliographystyle{apalike}
\author{Mateusz Dadej}
\date{\today}

\begin{document}

\maketitle

\subsection*{Introduction}

One of the main contributions of literature on financial networks is the property of financial system called robust-yet-fragile. The term was first coined ....


\subsection*{Theoretical background}

\subsection*{Research design}

\subsection*{Measures of financial connectedness}

Because of lack of transaction-level data among banks, the econometric literature provides several methods that aim at estimating a level of connectedness of financial sector. The measures are all estimated based on stock prices data, therefore providing a relatively high frequency variables. 

1. Average correlation 

\[\bar{\rho}(R) = \frac{\sum_{i \neq i}^{N} \sum_{j \neq j}^{N} \rho_{i,j}(R)}{N^2-N}\]

Where $\rho_{i,j}(R)$ is a correlation matrix of banks rate of returns.

2. 


Where the last two of them are as described in \citet{billio}

\subsection*{}

\bibliography{sample}


\end{document}