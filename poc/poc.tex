\documentclass{article}
\usepackage{algorithm}
\usepackage{algpseudocode}
\usepackage{natbib}
\usepackage{graphicx}
\usepackage{amsmath}
\usepackage{hyperref}
\hypersetup{
colorlinks=false,
pdfborder={0 0 0},
}
\title{Risk aversion heterogeneity and contagion in endogenous financial networks}
\bibliographystyle{apalike}
\author{Mateusz Dadej}
\date{\today}

\begin{document}

\maketitle


The following model builds on works of \citet{bluhm} and \citet{aldasoro}, it consists of $N$ banks, each balance sheet of bank $i$ satisfy the accounting identity:

\[c_i + \sum_{j=1}^{N}l_{i,j} + e_i = d_i + \sum_{j=1}^{N}b_{i,j} + q\]

Where left hand side are assets: $c$, $l$ and $e$ are cash, bank lendings and non-liquid assets respectively. Whereas, RHS consists of $d$, $b$ and $q$, that are deposits, bank borrowings and equity.

The interbank items $l_{i,j}$ and $b_{i,j}$  have indices showing the relation between banks, e.g. $l_{i,j}$ is a lending from bank $i$ to $j$. I denote aggregated values with single index: e.g., $l_{i} = \sum_{N}^{j=1} l_{i,j}$.

The profit function of bank depend on external $e$ and interbank $l$ assets and their returns $r^e_i$, $r^l$, default probability $\delta$, loss given default $\zeta$ and cost of borrowing $b_i$:

\[\pi_i = r_{i}^e e_i + r^l l_i - (\frac{1}{1 - \zeta \delta}) r^l b_i\]

The utility of a bank is a standard CRRA function, with its expected value derived with a taylor approximation (derivation in the appendix), giving the objective function:

\[\mathbf{E}[U(\pi, \sigma_i, \sigma_\pi)] = \frac{\pi^{1-\sigma_i}}{1 - \sigma_i} - \frac{\sigma_i}{2} \pi_i^{-1-\sigma_i} \sigma_\pi\]

Because of the expected value of utility function, the banks are optimizing risk-adjusted profits. The purpose of concave objective function (CRRA) is to impose liquidity hoarding among banks, depending on their risk aversion, which can also produce credit crunch in the system. Note the indices of $\sigma_i$, that represent the heterogeneity of each bank with respect to its risk aversion. This properity will be a subject to the analysis later on.

While deciding their optimal balance sheet, banks must also fulfills regulatory requirements:

\[\frac{c_i + l_i + e_i - d_i - b_i}{\omega_e e_i + \omega_l l_i} \geq (\gamma + \tau)\]

and

\[c_i \geq \alpha \times d_i\]

Where $\omega$ is weight of indexed asset class. $\gamma + \tau$ is regulatory requirement plus a safety margin of a bank. $\alpha$ is a share of deposits required to be held as cash.

\

Given above equations the optimization problem of bank $i$ is:

\[\max_{c_i, e_i, l_i, b_i} \mathbf{E}[U(\pi_i, \sigma_i, \sigma_\pi)]\]

S.t.:

\[c_i + l_i + e_i = d_i + b_i + q_i\]
\[\frac{c_i + e_i + l_i - d_i - b_i}{\omega_e e_i + \omega_l l_i} \geq (\gamma + \tau)\]
\[c_i \geq \alpha \times d_i\]

\subsection*{Interbank market}

As the objective function shows, the interbank market interest rate $r^l$ is given and does not vary for each bank. The banks decide their optimal asset and liability mix given the rate. An interbnak rate in equilibrium $\hat{r}^l$ is obtained through tatonnement process. Iteratively converging to it until the following balance of supply and demand is obtained:

\[\sum_{i}^{N} l_i^* = \sum_{i}^{N} b_i^*\]

Where $l_i^*$ and $b_i^*$ are optimal amount of lending and borrowing of each bank.

The algorithm adjust the interbank interest rate depending on the excess supply or demand until it converges to the equilibrium with the exact algorithm described below:

\begin{algorithm}
  \caption{tatonnement process}\label{alg:cap}
  \begin{algorithmic}
  \State $r^l \gets 0.05$
  \State $r^l_{max} \gets 0.1$
  \State $r^l_{min} \gets 0$
  \State $\Delta \gets \sum_{i}^{N} l_i^* - \sum_{i}^{N} b_i^*$
  \While{$|\Delta| < tol $}
  \If{$\Delta > 0$}
      \State $r^l_{max} \gets r^l$
      \State $r^l \gets \frac{r^l_{min} + r^l}{2}$
  \ElsIf{$\Delta < 0$}
      \State $r^l_{min} \gets r^l$
      \State $r^l \gets \frac{r^l_{max} + r^l}{2}$
  \EndIf
  \EndWhile
  \end{algorithmic}
\end{algorithm}

\newpage

In order to optimize the running time of the algorithm the tolerance for market balance might be adjusted. The excess market balance, altough economically negligible, is then split among banks deposit or cash.  

\subsection*{Matching algorithm}

The banks decide their aggregated levels of borrowing and lending given the equilibrium interest rate. However, the source of interconectedness are the exact links through the interbank markets. Next stage matches funds across the banks. Mathematically, it is a linear program that reconstruct matrix $A^{ib}$, where entry $A^{ib}_{i,j}$ is a value of loan from bank $i$ to $j$ (and vice versa). Thus, $\sum_{N}^{i=1} l_{i,j} = \sum_{N}^{j=1} A^{ib}_{i,j}$ (and vice versa for borrowings).

The banks match their aggregated supply and demand in a following optimization problem:

\begin{equation}
  \begin{aligned}
  \max_{A^{ib}_i} \quad & \sum_{i}^{N} \sigma_i (A^{ib}_i)^T k_i\\
  \textrm{s.t.} \quad & A^{ib}_{i,i} = 0 & \forall \; i \in N && \text{No self-lending}\\
    & A^{ib}_i \geq 0 & \forall \; i \in N && \text{No short-selling}\\
    & \sum_i^n A^{ib}_i = l_i & \forall \; i \in N && \text{Matching aggregated loans}\\
    & \sum_j^n A^{ib}_j = b_j & \forall \; j \in N && \text{Matching aggregated borrowing}\\ 
    & \frac{A^{ib}_i}{A^{\textbf{total}}_i} \leq \frac{1}{5} & \forall \; j \in N && \text{Maximum exposure limit}\\ 
  \end{aligned}
\end{equation}

Objective function maximizes sum of risk weighted assets  (weighted by the capital rate $k_i$) and risk aversion. This effectively fits more risky counterparties to less risk averse banks.

The constraints impose that the matrix $A^{ib}$ is greater or equal to zero (no short positions). The sum of rows is equal to aggregated loans (decided in optimization section). Analogically for borrowings. At last, not a single transaction can be higher than $20\%$ of balance sheet of lender (maximum exposure regulations).
    
\subsection*{Exogenous shock and contagion mechanics}

There is a initial shock to the system in equilibrium in the form of a single default of a random bank. The process of default is following:

\begin{itemize}
  \item Defaulted bank repays the deposits with cash holdings.
  \item If there are remaining deposits to repay, the bank is calling its loans from the interbank market.
  \item remainings from the called loans are distributed across interbank creditors.
  \item The rest of the interbank debt is written down.
  \item Creditors with defaulted interbank loans may have a negative equity and are thus defaulting themselves.
\end{itemize}

The contagion spreads further and stops when the system can finally absorb the losses. 

\subsection*{Simulation parameters}

\begin{center}
  \begin{tabular}{|c c|} 
   \hline
   Parameter & Value\\ [0.5ex] 
   \hline\hline
   $N$ & 20 \\ 
   \hline
   $\alpha$ & 0.01 \\
   \hline
   $\omega_e$ & 1 \\
   \hline
   $\omega_l$ & 0.2\\
   \hline
   $\tau$ & 0.01 \\ 
   \hline
   $\zeta$ & 0.6 \\ 
   \hline
   $E[\delta]$ & 0.05 \\
   \hline
   $\sigma_\delta$ & 0.03 \\ 
   \hline
   $r_e$ & $U(0.02, 0.15)$ \\ 
   \hline
   $V[r_e]$ & $\frac{1}{12}(\max(r^e) - \min(r^e))^2$ \\  [1ex] 
   \hline
  \end{tabular}
\end{center}

The data on deposits and equity of top 20 European banks in 2022 is taken from Orbis database.

\subsection*{Results}

We want to inspect the impact of risk aversion heterogeneity on financial contagion. Especially, if there is an effect analogical to the "super-spreaders" from epidemiology in the banking system.That is, what is the impact of a single extreme risk taker on the banking system. In order to do that, I have done following transformation on the risk aversion parameter:


\begin{equation}
  \begin{aligned}
    \sigma_{i \neq \textbf{ss}} = 2 + \frac{\Delta_{\textbf{ss}}}{N-1} \\ 
    \sigma_{\textbf{ss}} = 2 - \Delta_{\textbf{ss}}
  \end{aligned}
\end{equation}

The equation above decreases a risk aversion parameter of a single bank by $\Delta_{\textbf{ss}}$ and increases it among the rest of the banks, so that the expected value of risk aversion is the same. 

The model was simulated around 100 times for 4 different values of $\Delta_{\textbf{ss}}$ parameter:

\begin{table}[ht]
  \centering
  \begin{tabular}{rrrrrr}
    \hline
   & $\Delta_{\textbf{ss}}$ & mean \# defaults & st. deviation of defaults & P(\# defaults $>$ 2) & \# simulations \\ 
    \hline
 & 0 & 1.68 & 1.21 & 0.19 &  95 \\ 
 & 1 & 1.78 & 1.28 & 0.24 &  95 \\ 
 & 2 & 1.80 & 1.30 & 0.27 &  95 \\ 
 & 3 & 2.06 & 1.56 & 0.31 &  95 \\ 
     \hline
  \end{tabular}
\end{table}

The value $\Delta_{\textbf{ss}} = 0$ corresponds to the case of no "super spreader", i.e. every bank have the same risk aversion equal to 2. As the table above shows, each case with a single extreme risk taker makes contagion worse, either by increasing the severity (mean number of defaults) or by increasing the probability of serious contagion ($P(\textbf{\# defaults} > 2)$). Moreover, the parameter $\Delta_{\textbf{ss}}$ worsenes the contagion monotonically.

\subsection*{Conclusion}

Altough, the diagnostics and robustness checks of the model still needs to be improved, the results are very much promising and supports the evidence of the danger that extreme risk takers impose on the system. 

\bibliography{sample}

\subsection*{Appendix}

\textbf{Derivation of objective function}:

The following derivations draws from \citet{aldasoro}. Second order approximation of expected utility in the neighbourhood of expected value of profits is following:

\[U(\pi_i) \approx U(E[\pi_i]) + U'((\pi_i - E[\pi_i])) + \frac{1}{2} U''((\pi_i - E[\pi_i])^2)\]

Taking the expected value of both sides:

\begin{equation}
  \begin{aligned}
    E[U(\pi_i)] &\approx E[U(E[\pi_i])] + U'(E[(\pi_i - E[\pi_i])]) + \frac{1}{2} U''(E[(\pi_i - E[\pi_i])^2]) \\
     &\approx U(E[\pi_i]) + \frac{1}{2} U''(\pi_i) \sigma^2_{\pi}
  \end{aligned}
\end{equation}

\textbf{Derivation of profit variance}:

\begin{equation}
  \begin{aligned}
    \sigma^2_\pi &= V[r^e_i e_i + r^l l_i - \frac{1}{1 - \zeta \delta_i} r^l b_i] \\
    &= e_i^2 \sigma^2_{r^e_i} - (b_i r^l)^2 V[\frac{1}{1 - \zeta \delta_i}] + 2 e_i r^l \textbf{Cov}[r^e_i, \frac{1}{1 - \zeta \delta_i}]
  \end{aligned}
\end{equation}

Taking the first order Taylor approximation around expected value of $\delta_i$:

\[V[\frac{1}{1 - \zeta \delta_i}] = \zeta^2 (1 - \zeta E[\delta_i])^{-4} \sigma^2_{\delta_i}\]

\end{document}